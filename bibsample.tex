%% --------------------------------------------------------------------------
%% Compile with (PDF mode)
%%   pdflatex bibsample
%%   biber bibsample
%%   pdflatex bibsample
%%   pdflatex bibsample
%% or (DVI mode)
%%   latex bibsample
%%   biber bibsample
%%   latex bibsample
%%   latex bibsample
%% --------------------------------------------------------------------------
\documentclass[twoside]{nnw}
\usepackage[utf8]{inputenc}
\usepackage{amsmath,amssymb,amsfonts}
\usepackage[english]{babel}
%% Conditional support for PDF and DVI mode
\usepackage{ifpdf}
\ifpdf
  %% Needed for colouring links in PDF mode
  \usepackage{xcolor}
  \definecolor{cvutbase}{RGB}{0,99,168}
  \definecolor{cvutgreen}{RGB}{0,168,99}
  %% Can be used for electronic version
  \usepackage[pdftex,colorlinks=true,breaklinks=true,linkcolor=cvutbase,urlcolor=blue,citecolor=cvutgreen,plainpages=false]{hyperref}
\else
  %% B&W in DVI mode
  %% The unfortunate side effect of using DVI output is the lack of proper support
  %% for beaking long links (url, doi)
  \usepackage[dvips,colorlinks=true,breaklinks=true,linkcolor=black,urlcolor=black,citecolor=black]{hyperref}
  %% \nolinkurl is ignored by breakurl package, use a hack
  \def\nolinkurl#1{\url{#1}}
  %% This is needed in DVI->PS mode to break links
  %% \usepackage[anythingbreaks]{breakurl}
  \usepackage{breakurl}
\fi
%% BibTeX is hopelessly outdated to provide sensible rules for ISO690, and UTF-8
%% Hence, we depend on 'biblatex' package and 'biber' for .bib database processing
%% so instead of calling 'bibtex <file>' use 'biber <file>'
%% Biber part starts here
%\usepackage[
%    backend=biber,
%    sorting=nty,
%    useprefix=true,
%    style=nnw
%]{biblatex}
%% Biber part ends here
%% In the case of emergency you may still try to use BibTeX, but the result will probably
%% not be what you want. Biblatex will definitely complain that BibTeX is not UTF-8
%% capable and that some fields have different format (note that biblatex .bib format
%% uses and is able to parse more fields than bibtex .bib).
%% BibTeX part starts here
\usepackage[
    backend=bibtex,
    sorting=nty,
    useprefix=true,
    style=nnw
]{biblatex}
%% BibTeX part ends here
\renewcommand*{\bibfont}{\small}
\addbibresource{bibsample.bib}

\newcommand{\omath}[1]{%
  \ensuremath{#1}%
  \ifmmode \relax \else \xspace \fi%
}
\newcommand{\bi}[1]{\omath{\mathbf{#1}}}
\newcommand{\bs}[1]{\omath{\pmb{#1}}}
\newcommand{\tc}{\omath{T_\mathrm{C}}}
\newcommand{\tck}{\omath{T_{\mathrm{C},k}}}
\newcommand{\tz}{\omath{T_\mathrm{g}}}
\newcommand{\tzk}{\omath{T_{\mathrm{g},k}}}
\newcommand{\rpk}{\omath{R_{\mathrm{P},k}}}
\newcommand{\xak}{\omath{X_{\mathrm{a},k}}}
\newcommand{\kb}{\omath{a_{\mathrm{B},k}}}
\newcommand{\td}{\omath{T_\Delta}}
\newcommand{\doi}[1]{doi:~\href{http://dx.doi.org/#1}{\nolinkurl{#1}}}

% Provide citation commands used in natbib format
\makeatletter
\def\citep{\@ifnextchar[{\@with}{\@without}}
\def\citet{\@ifnextchar[{\@with}{\@without}}
\def\@with[#1]#2{\cite[#1]{#2}}
\def\@without#1{\cite{#1}}
\makeatother

\pagestyle{myheadings}
\setcounter{page}{1}
\markboth{Neural Network World ?/15, ??}{Přikryl, Kocijan: Modelling Occupancy-Queue Relation}

%\renewcommand{\baselinestretch}{2}

\begin{document}

\title{Modelling Occupancy-Queue Relation using Gaussian Process}

\author{%
    Jan Přikryl\thanks{Jan Přikryl -- Corresponding author, Czech Technical University in Prague, Faculty of Transportation Sciences, and Department of Adaptive Systems, Institute of Information Theory and Automation of the ASCR, Prague, Czech Republic. Email: \texttt{prikryl@fd.cvut.cz}}
    %
    and
    %
    Juš Kocijan\thanks{Juš Kocijan, Jozef Stefan Institute, Ljubljana and University of Nova Gorica, Slovenia}%
}

\maketitle

\begin{abstract}
%
One of the key indicators of the quality of service for urban transportation
control (UTC) systems is the queue length. Even in unsaturated conditions,
longer queues indicate longer travel delays and higher fuel consumption.
With the exception of some expensive surveillance equipment, the queue length
itself cannot be measured automatically, and manual measurement is both
impractical and costly in a long term scenario. Hence, many mathematical models
that express the queue length as a function of detector measurements are used
in engineering practice, ranging from simple to elaborate ones.  The method
proposed in this paper makes use of detector time-occupancy, a complementary
quantity to vehicle count, provided by most of the traffic detectors at no cost
and disregarded by majority of existing approaches for various reasons. Our model
is designed as a complement to existing methods. It is based on Gaussian-process
(GP)~model of the occupancy-queue relationship, it can handle data uncertainties,
and it provides more information about the quality of the queue length prediction.
%
\end{abstract}

\key{queue estimation; uncertainty; traffic model; Gaussian process}

\rec{August 28, 2014} \hfill  \textbf{DOI:} 10.14311/NNW.2015.25.00?
\rev{February 3, 2015}


% ========================================================================
%                                                             INTRODUCTION
% ========================================================================
\section{Introduction}

Queue length has been regarded as one of the key parameters in the process of signal plan design, as estimates of queue length may be used as a part of a criterion that is minimised by urban traffic control (UTC) systems that provide coordinated control of signalised intersections.

The main difficulty of using the queue-length as a part of performance criterion is the fact that the length is difficult to measure automatically -- the automatic systems are based exclusively on image processing \citep{yang2011robust,zheng2013measuring}.

Numerous studies discuss the problem of modelling the queue development, see for example \citet{friedrich03,mystkowski98,zuylen06,viloria00}. Typical queueing models are for example those of Akçelik \cite{akcelik80}, Hensher \cite{hensher00}, or Mück \cite{mueck02}. The American Highway Capacity Manual 2000 uses a modified version of Akçelik's model \cite{hcm2000}.  These models are derived from underlying physical principles of the queue formation and dissipation processes and include some ad-hoc corrections accounting for the stochastic nature of the queuing process.

Stochastic properties of queue development are directly taken into account by Markov chain models
%\citep{zuylen03,zuylen06,viti09a,viti09b}.
\cite{zuylen06,zuylen03,viti09a,viti2009dynamics}.
This class of models describes queueing as a stochastic process with probabilities of queue change being given by probability distributions.

The third class of models found in literature are black-box models trying to predict the queue length based on known ``training'' data. These include autoregressive models \citep{ho94}, neural networks \citep{chang95,ledoux97}, combination of neural networks and fuzzy logic \citep{quek06}, genetic algorithms \citep{girianna2004using}, or neural network constructed with the help of genetic algorithms \citep{vlahogianni05}.

All the models mentioned above compute queue length from vehicle count provided by upstream detectors. With the exception of \cite{diakaki99b} and publications of \cite{papageorgiou08a}
and \cite{vigos08a} (which all concentrate on estimating the total number of vehicles) and \cite{ma2012method} (which estimates the queue tail position by dense sampling the output signal of a detector loop, reading out the lengths of active pulses), detector time-occupancy is not used to provide additional information about the queue formation process, although this quantity is usually provided together with the vehicle count by an intersection controller.

In our opinion, the reason for disregarding time-occupancy information can be twofold: First, pure time-occupancy gives us a reasonable measure of queue length only for limited range of queue tails. If the distance from the detector to the downstream queue tail is high, the
time-occupancy stays low regardless of the queue length. If the queue tail reaches upstream far behind the detector, the time-occupancy will be high regardless of the queue length. Second,
the occupancy readings from the detector may be influenced by other parameters of the traffic flow, as inter-vehicle gaps, vehicle speeds, and their length.

Several papers appeared that try to make use of the time-occupancy measurements for providing additional information about the traffic state at an approach to a signalised intersection.
Authors of \citet{chang00} combine vehicle count and time-occupancy measurements with known signal state and a model of vehicle dynamics to estimate the queue length. Another approach has been taken in \cite{papageorgiou08a}, where the authors start with thorough theoretical analysis of the time-occupancy measurements and follow with a Kalman filter implementation of vehicle count estimator in the follow-up paper \citep{vigos08a}. In our publication \citep{prikryl09} we presented an empirical time-occupancy based queue-length model which combines two separate Gaussian-process (GP) models for the low and high occupancy ranges.

In this paper we present an empirical approach to queue length estimation from time-occupancy data, that is meant as a complement of the existing methods and that is able to provide estimation of queue length in the vicinity of a loop detector based on sparse occupancy measurements.

\printbibliography

\end{document}
